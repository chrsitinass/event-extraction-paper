\section{The Event Extraction Task}
%\subsection{Event Definition}
% 给出相关术语的定义,subsection名字起得对吗?
Event extraction aims to detect the occurrence of events with specific types and extract their typed participants or attributes from text. We clarify the following terminologies within our work:
\begin{itemize}
	\item \textbf{Event mention}: a phrase or sentence within which an event is described, including its type and arguments.
	\item \textbf{Argument}: an entity mention, temporal expression or value that is involved in an event, with specific roles.
	\item \textbf{Key argument}: the argument that plays an important role in one event, and helps to distinguish with other events. % 要不要举例说明什么是key argument?这里再举例会不会和introduction里面的举例重复?
	% 没有地方的时候,argument role可以删掉(我看别人ACE定义的时候都讲了这个就先放上来了)
	%\item \textbf{Argument role}: the relationship between an event and its involved argument. 
\end{itemize}

\subsection{Indirect Supervision for Event}%{Tabular Data}
% 缺一句连接的话?
We first utilize Freebase~\cite{bollacker2008freebase} as a source of supervision to guide our data construction, %the structured knowledge base.
% and there are three basics concepts: \emph{instance}, \emph{type} and \emph{property}. \emph{Instances} are entries in Freebase. \emph{Types} are different perspectives of \emph{instances}. 
where \textbf{\emph{Compound Value Type}} (CVT) is a special type to represent complex structured data % where entries are described 
with multiple \emph{properties}, usually organized in a table. Some CVT schemas indeed imply certain events, e.g., \emph{business.acquisition},
% \emph{military.military\_service} and \emph{people.marriage}, 
and closely resemble to event structures, where CVT properties can be treated as event arguments\footnote{
Therefore, we also use the term ``argument'' to refer to CVT property in the rest of paper.}. % 直接在这里说CVT property其实就是argument,后面描述的时候不会太咯嗦
As shown in Figure~\ref{fig:3}, the properties of CVT \emph{business.acquisition} actually can be used to label arguments of the events mentioned in S1 and S2. 
We use the Freebase copy of 2013-06,  %version of Berant et al. \shortcite{berant2013semantic}, 
containing 1010 CVTs. After manually filtering out those % CVTs that 
describing Freebase structure or irrelevant to events, %(e.g., \emph{food.recipe\_ingredient})
 we obtain 24 CVTs with around 280 million instances.

\begin{figure}[h]
	\centering
	\includegraphics[width=.48\textwidth]{temp_figure1.png}
	\caption{Examples of a CVT table in Freebase, and labeled sentences in our dataset. \emph{Company\_acquired}, \emph{acquiring\_company} and \emph{date} are key arguments in \emph{business.acquisition}. \label{fig:3}}
\end{figure}

Besides structured knowledge base, tables or lists that are used to sum up certain activities or occasions could be considered as a source of supervision for event extraction, as well. 
We thus investigate tables collected from Wikipedia pages, potentially referring to three types: winning of the Olympics, music and film awards, mergers and acquisitions\footnote{For example, \url{https://en.wikipedia.org/wiki/List_of_mergers_and_acquisitions_by_IBM}}.

\subsection{Dataset Construction\label{datagen}} 
% 在哪里说其实 property 在后面就是 argument 了
Here, we employ the event-related entries of Freebase CVT tables to illustrate how to automatically annotate event mentions in Wikipedia's articles, with the essence of distant supervision (\textbf{\texttt{DS}}): \textit{A sentence that contains \textbf{all key arguments} of an entry in an event table (e.g., CVT table) is likely to express that event}.
%\begin{quote}
%	\textbf{\texttt{DS}}: A sentence that contains all key arguments of an entry in an event table (e.g., CVT table) is likely to express that event in some way.
%\end{quote}
We will then label this sentence as a mention of this CVT event, and the words or phrases that match this entry's properties as the involved arguments, with the roles specified by their corresponding property names. 

We regard a sentence as \emph{positive} when it mentions the occurrence of an event, or  \emph{negative} otherwise. 
For example, S1 and S2 are positive examples with their arguments in italics and underlined (also shown in Figure~\ref{fig:3}), while S3 and S4 are negative.
%
\begin{quote}
	\textbf{S1}: \underline{\emph{Remedy Corp}} was sold to \underline{\emph{BMC Software}} as the \underline{\emph{Service}} \underline{\emph{Management Business Unit}} in \underline{\emph{2004}}.
\end{quote}
\begin{quote}
\textbf{S2}: \underline{\emph{Microsoft}} spent \$6.3 billion buying online display advertising company \underline{\emph{aQuantive}} in \underline{\emph{2007}}.
\end{quote}
\begin{quote}
\textbf{S3}: Microsoft hopes aQuantive's Brian McAndrews can outfox Google.
\end{quote}
\begin{quote}
\textbf{S4}: On April 29th, Elizabeth II and Prince Philip witnessed the marriage of Prince William.
\end{quote}

The selection strategy for key arguments for a given event type is based on two criteria: (1) \emph{Key arguments should have high importance value}; (2) \emph{Key arguments should include time-related arguments}.

% \paragraph{H1: Positive sentences should contain all properties}

% For example, S1 contains all the properties of instance $m.07bh4j7$ with a CVT type \emph{business.acquisition}, we thus consider S1 as a positive sample implying an event about \emph{business.acquisition}, and \emph{BMC Software}, \emph{Remedy Corp}, \emph{Service Management Business Unit} and \emph{2004} will be labeled as the arguments that play the role of \emph{acquiring\_company}, \emph{company\_acquired}, \emph{divisions\_formed}, and \emph{date} in this event, respectively.
% However, in practice, we realize that \emph{H1} is too strict that excludes a great many positive sentences like S2. 
% In practice, we find that \emph{H1} is too strict to include many positive sentences like S2. 
% We thus relax \emph{H1} by replacing \textbf{all properties} with \textbf{all key properties}. 

The \emph{Importance value} of an argument $arg$ (e.g., \emph{date}) to its event type $cvt$ (e.g., \emph{business.acquisition}) can be defined as:
\begin{equation}
	I_{cvt, arg} = log \frac{count(cvt, arg)}{count(cvt) \times count(arg)} 
\end{equation}
where $count(cvt)$ is the number of instances of type $cvt$, $count(arg)$ is the number of times $arg$ appearing in all CVT types, and $count(cvt, arg)$ is the number of $cvt$ instances that contain $arg$.

% We discover that for many CVTs, their key properties do not take into account time property. 
% time-related argument叫得对吗?本来的想说的是,取值为时间的argument
Although time-related arguments are often missing in the currently imperfect KBs, 
they are indeed crucial to indicate the actual occurrence of an event, e.g., S3, containing \emph{Microsoft} as \emph{acquiring\_company} and \emph{aQuantive} as \emph{company\_acquired} but without time-related arguments, will be mistakenly considered as a positive sample for event \emph{business.accquisition}.
% while contain all key properties of an instance, resulting in mistaking \emph{Microsoft} for \emph{acquiring\_company}, and \emph{aQuantive} for \emph{company\_acquired}. 
% By adding \emph{date} to the set of key properties, S3 will be filtered. 

Intuitively, two arguments involving in the same event mention are likely to be closer within the syntactic structure.
% , which will help to eliminate negative samples. 
In Figure~\ref{fig:2}, both \emph{Prince Philip} and \emph{marriage} can be matched as key arguments in a \textit{people.marriage} entry, but are far from each other on the dependency parse tree, thus S4 should be labeled as negative.
% We thus set the maximum distance between two key arguments as 2 empirically.
%, i.e., for a candidate sentence, if a pair of key arguments violates this constraint, it is supposed to be negative. 
% Given the dependency parsing tree in Figure~\ref{fig:2}, S4 is negative because the distance between \emph{Prince Philip} and \emph{marriage} is 3.

\begin{figure}
\centering
	\includegraphics[width=.47\textwidth]{temp_figure2}
	\caption{The dependency tree of S4, which partially matches an entry of \emph{people.marriage}. \label{fig:2}}
\end{figure}

We conduct a series of manual evaluations on the quantity and quality of the datasets produced by different strategies (see Sec~\ref{sec:evalhypo}), and 
% following strategy produces the best dataset, thus serves as 
our final strategy is: 
for each CVT, we first sort all its arguments in descending order by their importance values, and select the top half arguments as key arguments. 
We then include the time-related argument with highest importance value as a supplementary key argument. 
Finally, we eliminate sentences in which the dependency distances between any two key arguments are greater than 2.

% 好像没地方了,先不写了
% \begin{table}
% \centering
% \small
% \begin{tabular}{|l|l|} \hline
% CVT & Key arguments \\ \hline
% award.award\_honor & award\_winner, award, \ldots, year \\ \hline
% film.performance & actor, film, character \\ \hline
% education.education & institution, student, end\_date \\ \hline
% business.employment\_tenure & company, title, person, from \\ \hline
% \end{tabular}
% \caption{Examples of key arguments of four CVTs.\label{tab:5}}
% \end{table}

%\subsection{Our Task}
%Previous event extraction systems rely on explicit trigger identification to detect the occurrence of an event, 
%which is then used to decide its event type and label its arguments.
%In our automatically collected dataset, where human-labeled event triggers are unavailable, we argue that \textbf{key arguments} can play the same role as explicit event triggers. 
%We thus treat the event extraction as a pipeline of the following two steps: 
%\begin{itemize}
%	\item \textbf{Event detection}: to identify key arguments in a sentence. If a sentence contains \textbf{all key arguments} of a specific event type, it will be considered to imply an event mention of this specified type. 
%	\item \textbf{Argument detection}: to identify other non-key arguments for each event in the sentence.
%\end{itemize}
%
%Take S1 as an example, in event detection, \emph{Remedy Corp}, \emph{BMC Software}, and \emph{2004} could be identified as \emph{company\_acquired}, \emph{acquiring\_company}, and \emph{date}, respectively, indicating that S1 may mention a \emph{business.acquisition} event. 
%During argument detection, \emph{Service Management Business Unit} should be identified as \emph{divisions\_formed}, 
%which, together with the detected key arguments, form a full mention for a \emph{business.acquisition} event.  
