\begin{abstract}
Existing event extraction systems are typically investigated in a supervised
learning paradigm, which heavily relies on the quality of
expert-annotated datasets, % which still require a costly and time-consuming process to construct.
% As a result, the built datasets
usually covering a limited variety of event types and making the learned event extractor hard to generalize. 
In this paper, we address the problem
of automatically building event extractors for rich event types with little expert involvement.
We achieve this by employing distant supervision to automatically create event annotations from
unlabelled text using structured knowledge bases or tables. We propose a novel
neural network model with post inference, to detect multi-typed event mentions with corresponding arguments.
%We evaluate our approach by investigating the feasibility of  automatically collecting training data for event extraction from both 
Experiments on the datasets collected through Freebase and Wikipedia tables show
that 
%our proposed extraction model is designed to identify both typed event mentions and typed arguments. 
%Both automatic and manual evaluations demonstrate that 
it is possible to learn to extract events of rich types without human-annotated training data.

%
%and rely on expert-annotated datasets, with limited event types.
%%such as ACE and ERE event extraction frameworks.
%However, designing and constructing these
%high-quality corpora, usually with limited size and coverage of event types,  is costly, which
%makes learned extractors hard to generalize.  With the essence of distant supervision,
%%Inspired by some Freebase schemas which share similar structures with ACE event templates,
%we investigate the possibilities of automatic construction of training data for various event types
%with the help of structured knowledge bases.
%the following problems in this paper: can we generate a feasible dataset for event extraction with Freebase automatically and is it possible to extract events on this dataset.
%We first propose four hypotheses based on our observation and produce our dataset accordingly. Then,
%We further propose a novel neural network with ILP-based post inference, committing to
%handling two challenges in event extraction: multi-type events and multi-word arguments.
%Both automatic and manual evaluations demonstrate that it is possible to learn to extract various  events, according to existing knowledge bases, without human-annotated training data.
\end{abstract}
